\documentclass[]{book}
\usepackage{lmodern}
\usepackage{amssymb,amsmath}
\usepackage{ifxetex,ifluatex}
\usepackage{fixltx2e} % provides \textsubscript
\ifnum 0\ifxetex 1\fi\ifluatex 1\fi=0 % if pdftex
  \usepackage[T1]{fontenc}
  \usepackage[utf8]{inputenc}
\else % if luatex or xelatex
  \ifxetex
    \usepackage{mathspec}
  \else
    \usepackage{fontspec}
  \fi
  \defaultfontfeatures{Ligatures=TeX,Scale=MatchLowercase}
\fi
% use upquote if available, for straight quotes in verbatim environments
\IfFileExists{upquote.sty}{\usepackage{upquote}}{}
% use microtype if available
\IfFileExists{microtype.sty}{%
\usepackage{microtype}
\UseMicrotypeSet[protrusion]{basicmath} % disable protrusion for tt fonts
}{}
\usepackage[margin=1in]{geometry}
\usepackage{hyperref}
\hypersetup{unicode=true,
            pdftitle={Modeling Melodic Dictation},
            pdfauthor={David John Baker},
            pdfborder={0 0 0},
            breaklinks=true}
\urlstyle{same}  % don't use monospace font for urls
\usepackage{natbib}
\bibliographystyle{apalike}
\usepackage{color}
\usepackage{fancyvrb}
\newcommand{\VerbBar}{|}
\newcommand{\VERB}{\Verb[commandchars=\\\{\}]}
\DefineVerbatimEnvironment{Highlighting}{Verbatim}{commandchars=\\\{\}}
% Add ',fontsize=\small' for more characters per line
\usepackage{framed}
\definecolor{shadecolor}{RGB}{248,248,248}
\newenvironment{Shaded}{\begin{snugshade}}{\end{snugshade}}
\newcommand{\AlertTok}[1]{\textcolor[rgb]{0.94,0.16,0.16}{#1}}
\newcommand{\AnnotationTok}[1]{\textcolor[rgb]{0.56,0.35,0.01}{\textbf{\textit{#1}}}}
\newcommand{\AttributeTok}[1]{\textcolor[rgb]{0.77,0.63,0.00}{#1}}
\newcommand{\BaseNTok}[1]{\textcolor[rgb]{0.00,0.00,0.81}{#1}}
\newcommand{\BuiltInTok}[1]{#1}
\newcommand{\CharTok}[1]{\textcolor[rgb]{0.31,0.60,0.02}{#1}}
\newcommand{\CommentTok}[1]{\textcolor[rgb]{0.56,0.35,0.01}{\textit{#1}}}
\newcommand{\CommentVarTok}[1]{\textcolor[rgb]{0.56,0.35,0.01}{\textbf{\textit{#1}}}}
\newcommand{\ConstantTok}[1]{\textcolor[rgb]{0.00,0.00,0.00}{#1}}
\newcommand{\ControlFlowTok}[1]{\textcolor[rgb]{0.13,0.29,0.53}{\textbf{#1}}}
\newcommand{\DataTypeTok}[1]{\textcolor[rgb]{0.13,0.29,0.53}{#1}}
\newcommand{\DecValTok}[1]{\textcolor[rgb]{0.00,0.00,0.81}{#1}}
\newcommand{\DocumentationTok}[1]{\textcolor[rgb]{0.56,0.35,0.01}{\textbf{\textit{#1}}}}
\newcommand{\ErrorTok}[1]{\textcolor[rgb]{0.64,0.00,0.00}{\textbf{#1}}}
\newcommand{\ExtensionTok}[1]{#1}
\newcommand{\FloatTok}[1]{\textcolor[rgb]{0.00,0.00,0.81}{#1}}
\newcommand{\FunctionTok}[1]{\textcolor[rgb]{0.00,0.00,0.00}{#1}}
\newcommand{\ImportTok}[1]{#1}
\newcommand{\InformationTok}[1]{\textcolor[rgb]{0.56,0.35,0.01}{\textbf{\textit{#1}}}}
\newcommand{\KeywordTok}[1]{\textcolor[rgb]{0.13,0.29,0.53}{\textbf{#1}}}
\newcommand{\NormalTok}[1]{#1}
\newcommand{\OperatorTok}[1]{\textcolor[rgb]{0.81,0.36,0.00}{\textbf{#1}}}
\newcommand{\OtherTok}[1]{\textcolor[rgb]{0.56,0.35,0.01}{#1}}
\newcommand{\PreprocessorTok}[1]{\textcolor[rgb]{0.56,0.35,0.01}{\textit{#1}}}
\newcommand{\RegionMarkerTok}[1]{#1}
\newcommand{\SpecialCharTok}[1]{\textcolor[rgb]{0.00,0.00,0.00}{#1}}
\newcommand{\SpecialStringTok}[1]{\textcolor[rgb]{0.31,0.60,0.02}{#1}}
\newcommand{\StringTok}[1]{\textcolor[rgb]{0.31,0.60,0.02}{#1}}
\newcommand{\VariableTok}[1]{\textcolor[rgb]{0.00,0.00,0.00}{#1}}
\newcommand{\VerbatimStringTok}[1]{\textcolor[rgb]{0.31,0.60,0.02}{#1}}
\newcommand{\WarningTok}[1]{\textcolor[rgb]{0.56,0.35,0.01}{\textbf{\textit{#1}}}}
\usepackage{longtable,booktabs}
\usepackage{graphicx,grffile}
\makeatletter
\def\maxwidth{\ifdim\Gin@nat@width>\linewidth\linewidth\else\Gin@nat@width\fi}
\def\maxheight{\ifdim\Gin@nat@height>\textheight\textheight\else\Gin@nat@height\fi}
\makeatother
% Scale images if necessary, so that they will not overflow the page
% margins by default, and it is still possible to overwrite the defaults
% using explicit options in \includegraphics[width, height, ...]{}
\setkeys{Gin}{width=\maxwidth,height=\maxheight,keepaspectratio}
\IfFileExists{parskip.sty}{%
\usepackage{parskip}
}{% else
\setlength{\parindent}{0pt}
\setlength{\parskip}{6pt plus 2pt minus 1pt}
}
\setlength{\emergencystretch}{3em}  % prevent overfull lines
\providecommand{\tightlist}{%
  \setlength{\itemsep}{0pt}\setlength{\parskip}{0pt}}
\setcounter{secnumdepth}{5}
% Redefines (sub)paragraphs to behave more like sections
\ifx\paragraph\undefined\else
\let\oldparagraph\paragraph
\renewcommand{\paragraph}[1]{\oldparagraph{#1}\mbox{}}
\fi
\ifx\subparagraph\undefined\else
\let\oldsubparagraph\subparagraph
\renewcommand{\subparagraph}[1]{\oldsubparagraph{#1}\mbox{}}
\fi

%%% Use protect on footnotes to avoid problems with footnotes in titles
\let\rmarkdownfootnote\footnote%
\def\footnote{\protect\rmarkdownfootnote}

%%% Change title format to be more compact
\usepackage{titling}

% Create subtitle command for use in maketitle
\newcommand{\subtitle}[1]{
  \posttitle{
    \begin{center}\large#1\end{center}
    }
}

\setlength{\droptitle}{-2em}
  \title{Modeling Melodic Dictation}
  \pretitle{\vspace{\droptitle}\centering\huge}
  \posttitle{\par}
  \author{David John Baker}
  \preauthor{\centering\large\emph}
  \postauthor{\par}
  \predate{\centering\large\emph}
  \postdate{\par}
  \date{2018-08-02}

\usepackage{booktabs}
\usepackage{amsthm}
\makeatletter
\def\thm@space@setup{%
  \thm@preskip=8pt plus 2pt minus 4pt
  \thm@postskip=\thm@preskip
}
\makeatother

\usepackage{amsthm}
\newtheorem{theorem}{Theorem}[chapter]
\newtheorem{lemma}{Lemma}[chapter]
\theoremstyle{definition}
\newtheorem{definition}{Definition}[chapter]
\newtheorem{corollary}{Corollary}[chapter]
\newtheorem{proposition}{Proposition}[chapter]
\theoremstyle{definition}
\newtheorem{example}{Example}[chapter]
\theoremstyle{definition}
\newtheorem{exercise}{Exercise}[chapter]
\theoremstyle{remark}
\newtheorem*{remark}{Remark}
\newtheorem*{solution}{Solution}
\begin{document}
\maketitle

{
\setcounter{tocdepth}{1}
\tableofcontents
}
\hypertarget{prerequisites}{%
\chapter{Prerequisites}\label{prerequisites}}

This is a \emph{sample} book written in \textbf{Markdown}. You can use
anything that Pandoc's Markdown supports, e.g., a math equation
\(a^2 + b^2 = c^2\).

The \textbf{bookdown} package can be installed from CRAN or Github:

\begin{Shaded}
\begin{Highlighting}[]
\KeywordTok{install.packages}\NormalTok{(}\StringTok{"bookdown"}\NormalTok{)}
\CommentTok{# or the development version}
\CommentTok{# devtools::install_github("rstudio/bookdown")}
\end{Highlighting}
\end{Shaded}

Remember each Rmd file contains one and only one chapter, and a chapter
is defined by the first-level heading \texttt{\#}.

To compile this example to PDF, you need XeLaTeX. You are recommended to
install TinyTeX (which includes XeLaTeX):
\url{https://yihui.name/tinytex/}.

\hypertarget{intro}{%
\chapter{Introduction}\label{intro}}

\hypertarget{prospectus-inclusion}{%
\section{Prospectus Inclusion}\label{prospectus-inclusion}}

Note that there are two fields, both of which's literature can help out.

\hypertarget{theoretical-background}{%
\subsection{Theoretical Background}\label{theoretical-background}}

\hypertarget{computational-musicology}{%
\subsubsection{Computational
Musicology}\label{computational-musicology}}

\hypertarget{music-psychology-and-memory-for-melody}{%
\subsubsection{Music Psychology and Memory for
Melody}\label{music-psychology-and-memory-for-melody}}

\hypertarget{rationale}{%
\subsection{Rationale}\label{rationale}}

\hypertarget{computational-musicology-1}{%
\subsubsection{Computational
Musicology}\label{computational-musicology-1}}

\hypertarget{music-psychology}{%
\subsubsection{Music Psychology}\label{music-psychology}}

\hypertarget{factors}{%
\subsection{Factors}\label{factors}}

This section will list factors that are believed to be important to
modeling melodic dictation. Need to have both individual and musical
parameters. Ends with polymorphic view of musicianship.

\hypertarget{history-of-aural-skills}{%
\chapter{History of Aural Skills}\label{history-of-aural-skills}}

\hypertarget{history-of-aural}{%
\section{History of Aural}\label{history-of-aural}}

\begin{itemize}
\tightlist
\item
  Compare and contrast goals in terms of pedagogy and teaching.
\end{itemize}

\hypertarget{current-state}{%
\section{Current State}\label{current-state}}

\begin{itemize}
\tightlist
\item
  Books and what not.
\end{itemize}

\hypertarget{individual-differences}{%
\chapter{Individual Differences}\label{individual-differences}}

\hypertarget{cognitive-aparatus}{%
\section{Cognitive Aparatus}\label{cognitive-aparatus}}

\hypertarget{training-effects}{%
\section{Training Effects}\label{training-effects}}

\hypertarget{transfere-literature}{%
\section{Transfere Literature}\label{transfere-literature}}

\hypertarget{memory-for-melodies-literature}{%
\section{Memory for Melodies
Literature}\label{memory-for-melodies-literature}}

\hypertarget{musical-parameters}{%
\chapter{Musical Parameters}\label{musical-parameters}}

\hypertarget{inspiration-from-computational-linguistics}{%
\section{Inspiration from Computational
Linguistics}\label{inspiration-from-computational-linguistics}}

\hypertarget{feature-extraction-in-music}{%
\section{Feature Extraction in
Music}\label{feature-extraction-in-music}}

\hypertarget{symbolic-approaches-static}{%
\subsection{Symbolic Approaches
(Static)}\label{symbolic-approaches-static}}

\hypertarget{symbolic-appoaches-dynamic}{%
\subsection{Symbolic Appoaches
(Dynamic)}\label{symbolic-appoaches-dynamic}}

\hypertarget{behavioral-results}{%
\subsection{Behavioral Results}\label{behavioral-results}}

\hypertarget{point-is-that-these-features-can-stand-in-for-intuition}{%
\section{Point is that these features can stand in for
intuition}\label{point-is-that-these-features-can-stand-in-for-intuition}}

\hypertarget{corpus}{%
\chapter{Corpus}\label{corpus}}

\hypertarget{why-need-new-data}{%
\section{Why need new data}\label{why-need-new-data}}

\hypertarget{history-of-corpus-studies}{%
\section{History of Corpus Studies}\label{history-of-corpus-studies}}

\hypertarget{current-state-in-music}{%
\section{Current State in Music}\label{current-state-in-music}}

\hypertarget{limitations}{%
\section{Limitations}\label{limitations}}

\hypertarget{boring-corpus-stuff}{%
\section{Boring Corpus Stuff}\label{boring-corpus-stuff}}

\hypertarget{encoding-process}{%
\subsection{Encoding Process}\label{encoding-process}}

\hypertarget{sampling-criteria}{%
\subsection{Sampling Criteria}\label{sampling-criteria}}

\hypertarget{situation-of-corpus-methods}{%
\subsection{Situation of Corpus
Methods}\label{situation-of-corpus-methods}}

\hypertarget{descriptives-of-the-corpus-compared-to-essendutchwhatever}{%
\section{Descriptives of the Corpus compared to
Essen/Dutch/Whatever}\label{descriptives-of-the-corpus-compared-to-essendutchwhatever}}

\hypertarget{final-words}{%
\chapter{Final Words}\label{final-words}}

We have finished a nice book.

You can label chapter and section titles using \texttt{\{\#label\}}
after them, e.g., we can reference Chapter \ref{intro}. If you do not
manually label them, there will be automatic labels anyway, e.g.,
Chapter \ref{methods}.

Figures and tables with captions will be placed in \texttt{figure} and
\texttt{table} environments, respectively.

\begin{Shaded}
\begin{Highlighting}[]
\KeywordTok{par}\NormalTok{(}\DataTypeTok{mar =} \KeywordTok{c}\NormalTok{(}\DecValTok{4}\NormalTok{, }\DecValTok{4}\NormalTok{, }\FloatTok{.1}\NormalTok{, }\FloatTok{.1}\NormalTok{))}
\KeywordTok{plot}\NormalTok{(pressure, }\DataTypeTok{type =} \StringTok{'b'}\NormalTok{, }\DataTypeTok{pch =} \DecValTok{19}\NormalTok{)}
\end{Highlighting}
\end{Shaded}

\begin{figure}

{\centering \includegraphics[width=0.8\linewidth]{bookdown-demo_files/figure-latex/nice-fig-1} 

}

\caption{Here is a nice figure!}\label{fig:nice-fig}
\end{figure}

Reference a figure by its code chunk label with the \texttt{fig:}
prefix, e.g., see Figure \ref{fig:nice-fig}. Similarly, you can
reference tables generated from \texttt{knitr::kable()}, e.g., see Table
\ref{tab:nice-tab}.

\begin{Shaded}
\begin{Highlighting}[]
\NormalTok{knitr}\OperatorTok{::}\KeywordTok{kable}\NormalTok{(}
  \KeywordTok{head}\NormalTok{(iris, }\DecValTok{20}\NormalTok{), }\DataTypeTok{caption =} \StringTok{'Here is a nice table!'}\NormalTok{,}
  \DataTypeTok{booktabs =} \OtherTok{TRUE}
\NormalTok{)}
\end{Highlighting}
\end{Shaded}

\begin{table}

\caption{\label{tab:nice-tab}Here is a nice table!}
\centering
\begin{tabular}[t]{rrrrl}
\toprule
Sepal.Length & Sepal.Width & Petal.Length & Petal.Width & Species\\
\midrule
5.1 & 3.5 & 1.4 & 0.2 & setosa\\
4.9 & 3.0 & 1.4 & 0.2 & setosa\\
4.7 & 3.2 & 1.3 & 0.2 & setosa\\
4.6 & 3.1 & 1.5 & 0.2 & setosa\\
5.0 & 3.6 & 1.4 & 0.2 & setosa\\
\addlinespace
5.4 & 3.9 & 1.7 & 0.4 & setosa\\
4.6 & 3.4 & 1.4 & 0.3 & setosa\\
5.0 & 3.4 & 1.5 & 0.2 & setosa\\
4.4 & 2.9 & 1.4 & 0.2 & setosa\\
4.9 & 3.1 & 1.5 & 0.1 & setosa\\
\addlinespace
5.4 & 3.7 & 1.5 & 0.2 & setosa\\
4.8 & 3.4 & 1.6 & 0.2 & setosa\\
4.8 & 3.0 & 1.4 & 0.1 & setosa\\
4.3 & 3.0 & 1.1 & 0.1 & setosa\\
5.8 & 4.0 & 1.2 & 0.2 & setosa\\
\addlinespace
5.7 & 4.4 & 1.5 & 0.4 & setosa\\
5.4 & 3.9 & 1.3 & 0.4 & setosa\\
5.1 & 3.5 & 1.4 & 0.3 & setosa\\
5.7 & 3.8 & 1.7 & 0.3 & setosa\\
5.1 & 3.8 & 1.5 & 0.3 & setosa\\
\bottomrule
\end{tabular}
\end{table}

You can write citations, too. For example, we are using the
\textbf{bookdown} package \citep{R-bookdown} in this sample book, which
was built on top of R Markdown and \textbf{knitr} \citep{xie2015}.

\hypertarget{experiments}{%
\chapter{Experiments}\label{experiments}}

\hypertarget{rationale-for-experiment}{%
\section{Rationale for Experiment}\label{rationale-for-experiment}}

\hypertarget{selection-of-melodies}{%
\section{Selection of Melodies}\label{selection-of-melodies}}

\hypertarget{experiment-i-and-ii}{%
\section{Experiment I and II}\label{experiment-i-and-ii}}

\hypertarget{experiment-iii}{%
\section{Experiment III?}\label{experiment-iii}}

\hypertarget{limitations-1}{%
\section{Limitations}\label{limitations-1}}

\hypertarget{how-to-score}{%
\subsection{How to Score}\label{how-to-score}}

\hypertarget{reasons-for-making-everything-open-source}{%
\subsection{Reasons for making everything open
source}\label{reasons-for-making-everything-open-source}}

\hypertarget{summaries}{%
\section{Summaries}\label{summaries}}

\hypertarget{applications-to-pedagoges}{%
\subsection{Applications to Pedagoges}\label{applications-to-pedagoges}}

\hypertarget{conceptual-frameworks}{%
\subsection{Conceptual Frameworks}\label{conceptual-frameworks}}

\hypertarget{conclusions}{%
\section{Conclusions}\label{conclusions}}

\hypertarget{what-can-we-really-expect-of-undergrads}{%
\subsection{What can we really expect of
undergrads?}\label{what-can-we-really-expect-of-undergrads}}

\bibliography{book.bib,packages.bib}


\end{document}
